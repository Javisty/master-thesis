%%%%%%%%%%%%%%%%%%%%%%%%%%%%%%%%%%%%%%%%%
% Masters/Doctoral Thesis 
% LaTeX Template
% Version 2.5 (27/8/17)
%
% This template was downloaded from:
% http://www.LaTeXTemplates.com
%
% Version 2.x major modifications by:
% Vel (vel@latextemplates.com)
%
% This template is based on a template by:
% Steve Gunn (http://users.ecs.soton.ac.uk/srg/softwaretools/document/templates/)
% Sunil Patel (http://www.sunilpatel.co.uk/thesis-template/)
%
% Template license:
% CC BY-NC-SA 3.0 (http://creativecommons.org/licenses/by-nc-sa/3.0/)
%
%%%%%%%%%%%%%%%%%%%%%%%%%%%%%%%%%%%%%%%%%

%----------------------------------------------------------------------------------------
%	PACKAGES AND OTHER DOCUMENT CONFIGURATIONS
%----------------------------------------------------------------------------------------

\documentclass[
11pt, % The default document font size, options: 10pt, 11pt, 12pt
%oneside, % Two side (alternating margins) for binding by default, uncomment to switch to one side
english, % ngerman for German
singlespacing, % Single line spacing, alternatives: onehalfspacing or doublespacing
%draft, % Uncomment to enable draft mode (no pictures, no links, overfull hboxes indicated)
%nolistspacing, % If the document is onehalfspacing or doublespacing, uncomment this to set spacing in lists to single
%liststotoc, % Uncomment to add the list of figures/tables/etc to the table of contents
%toctotoc, % Uncomment to add the main table of contents to the table of contents
%parskip, % Uncomment to add space between paragraphs
%nohyperref, % Uncomment to not load the hyperref package
headsepline, % Uncomment to get a line under the header
%chapterinoneline, % Uncomment to place the chapter title next to the number on one line
%consistentlayout, % Uncomment to change the layout of the declaration, abstract and acknowledgements pages to match the default layout
]{MastersDoctoralThesis} % The class file specifying the document structure

\usepackage[utf8]{inputenc} % Required for inputting international characters
\usepackage[T1]{fontenc} % Output font encoding for international characters

\usepackage{mathpazo} % Use the Palatino font by default

\usepackage[backend=bibtex,style=authoryear,natbib=true]{biblatex} % Use the bibtex backend with the authoryear citation style (which resembles APA)

\addbibresource{hierarchical.bib} % The filename of the bibliography
\addbibresource{reinforcement_learning.bib}

\usepackage[autostyle=true]{csquotes} % Required to generate language-dependent quotes in the bibliography

\usepackage{graphicx} %Loading the package
\graphicspath{{Figures/}{Logos/}} %Setting the graphicspath

%----------------------------------------------------------------------------------------
%	MARGIN SETTINGS
%----------------------------------------------------------------------------------------

\geometry{
	paper=a4paper, % Change to letterpaper for US letter
	inner=2.5cm, % Inner margin
	outer=3.8cm, % Outer margin
	bindingoffset=.5cm, % Binding offset
	top=1.5cm, % Top margin
	bottom=1.5cm, % Bottom margin
	%showframe, % Uncomment to show how the type block is set on the page
}

%----------------------------------------------------------------------------------------
%       MISCELLANEOUS SETTINGS
%----------------------------------------------------------------------------------------

\usepackage{amsmath}
\usepackage{amsthm}

\usepackage{algorithm}%
\usepackage{algorithmicx}%
\usepackage{algpseudocode}%

\newtheorem{thm}{Theorem}[section]
\newtheorem{lem}{Lemma}[section]
\newtheorem{prop}{Proposition}[section]
\newtheorem{defi}{Definition}[section]

%----------------------------------------------------------------------------------------
%	THESIS INFORMATION
%----------------------------------------------------------------------------------------

\thesistitle{Model-free Reinforcement Learning with Temporal Abstraction} % Your thesis title, this is used in the title and abstract, print it elsewhere with \ttitle
\supervisor{Dr. Sadegh \textsc{Talebi}} % Your supervisor's name, this is used in the title page, print it elsewhere with \supname
\examiner{} % Your examiner's name, this is not currently used anywhere in the template, print it elsewhere with \examname
\degree{Master of Data Science} % Your degree name, this is used in the title page and abstract, print it elsewhere with \degreename
\author{Aymeric \textsc{Côme}} % Your name, this is used in the title page and abstract, print it elsewhere with \authorname
\addresses{} % Your address, this is not currently used anywhere in the template, print it elsewhere with \addressname

\subject{Computer Science} % Your subject area, this is not currently used anywhere in the template, print it elsewhere with \subjectname
\keywords{Hierarchical Reinforcement learning, Average-reward MDP, Options} % Keywords for your thesis, this is not currently used anywhere in the template, print it elsewhere with \keywordnames
\university{\href{https://sciences-technologies.univ-lille.fr/informatique/formation/master-data-science/}{University of Lille - Centrale Lille - IMT Lille-Douai}} % Your university's name and URL, this is used in the title page and abstract, print it elsewhere with \univname
\department{\href{https://di.ku.dk/english/}{Department of Computer Science}} % Your department's name and URL, this is used in the title page and abstract, print it elsewhere with \deptname
\group{\href{https://sites.google.com/diku.edu/delta}{DeLTA research group}} % Your research group's name and URL, this is used in the title page, print it elsewhere with \groupname
\faculty{\href{https://www.ku.dk/english/}{University of Copenhagen}} % Your faculty's name and URL, this is used in the title page and abstract, print it elsewhere with \facname

\AtBeginDocument{
\hypersetup{pdftitle=\ttitle} % Set the PDF's title to your title
\hypersetup{pdfauthor=\authorname} % Set the PDF's author to your name
\hypersetup{pdfkeywords=\keywordnames} % Set the PDF's keywords to your keywords
}

\begin{document}

\frontmatter % Use roman page numbering style (i, ii, iii, iv...) for the pre-content pages

\pagestyle{plain} % Default to the plain heading style until the thesis style is called for the body content

%----------------------------------------------------------------------------------------
%	TITLE PAGE
%----------------------------------------------------------------------------------------

\begin{titlepage}
\begin{center}

\vspace*{.00\textheight}
        {\scshape\large \univname\par}\vspace{0.4cm} % University name
        \begin{figure}
          \centering
          \includegraphics[width=0.3\textwidth]{ulille_logo_noir.png}
          \includegraphics[width=0.3\textwidth]{Centrale_Lille.png} % University/department logo - uncomment to place it
          \includegraphics[width=0.3\textwidth]{IMT_Lille_douai_Logo_WEB.png}
        \end{figure}

\textsc{\Large Master Thesis}\\[0.5cm] % Thesis type

\HRule \\[0.4cm] % Horizontal line
{\huge \bfseries \ttitle\par}\vspace{0.4cm} % Thesis title
\HRule \\[1.5cm] % Horizontal line
 
\begin{minipage}[t]{0.4\textwidth}
\begin{flushleft} \large
\emph{Author:}\\
\authorname % Author name - remove the \href bracket to remove the link
\end{flushleft}
\end{minipage}
\begin{minipage}[t]{0.4\textwidth}
\begin{flushright} \large
\emph{Supervisor:} \\
\href{https://sites.google.com/view/talebi/}{\supname} % Supervisor name - remove the \href bracket to remove the link  
\end{flushright}
\end{minipage}\\[2cm]
 
\vfill

\large \textit{A thesis submitted in fulfillment of the requirements\\ for the degree of \degreename}\\[0.3cm] % University requirement text
\textit{in the}\\[0.4cm]
\groupname\\\deptname\\\facname\\[1cm] % Research group name and department name
\includegraphics[width=0.8\textwidth, height=5\baselineskip, keepaspectratio=true]{university-of-copenhagen.png}

\vfill

{\large \today}\\[1cm] % Date
 
\vfill
\end{center}
\end{titlepage}

%----------------------------------------------------------------------------------------
%	DECLARATION PAGE
%----------------------------------------------------------------------------------------

%% \begin{declaration}
%% \addchaptertocentry{\authorshipname} % Add the declaration to the table of contents
%% \noindent I, \authorname, declare that this thesis titled, \enquote{\ttitle} and the work presented in it are my own. I confirm that:

%% \begin{itemize} 
%% \item This work was done wholly or mainly while in candidature for a research degree at this University.
%% \item Where any part of this thesis has previously been submitted for a degree or any other qualification at this University or any other institution, this has been clearly stated.
%% \item Where I have consulted the published work of others, this is always clearly attributed.
%% \item Where I have quoted from the work of others, the source is always given. With the exception of such quotations, this thesis is entirely my own work.
%% \item I have acknowledged all main sources of help.
%% \item Where the thesis is based on work done by myself jointly with others, I have made clear exactly what was done by others and what I have contributed myself.\\
%% \end{itemize}
 
%% \noindent Signed:\\
%% \rule[0.5em]{25em}{0.5pt} % This prints a line for the signature
 
%% \noindent Date:\\
%% \rule[0.5em]{25em}{0.5pt} % This prints a line to write the date
%% \end{declaration}

%% \cleardoublepage

%----------------------------------------------------------------------------------------
%	ABSTRACT PAGE
%----------------------------------------------------------------------------------------

\begin{abstract}
\addchaptertocentry{\abstractname} % Add the abstract to the table of contents
Control through Reinforcement Learning aims at choosing good strategies under uncertainty; however, because of the latter, identifying long-term profitable actions is often challenging, and requires longer training time. The acknowledgment of a pattern in the environment on the other hand might alleviate the amount of experience needed to make a confident decision, as then the problem could be decomposed into a hierarchy of sub-tasks. In particular, reasoning at different time scales can significantly help thinking long-term. Yet, while the literature features many approaches leveraging such temporal abstraction, and experiments show their successes, the theoretical benefits or downsides are still hardly understood. This thesis attempts to provide theoretical guarantees for a basic algorithm, as a step towards a better understanding of temporal abstraction in reinforcement learning.
\end{abstract}

%----------------------------------------------------------------------------------------
%	ACKNOWLEDGEMENTS
%----------------------------------------------------------------------------------------

\begin{acknowledgements}
\addchaptertocentry{\acknowledgementname} % Add the acknowledgements to the table of contents
I would like to thank anyone that has allowed me to udnertake this thesis in Denmark, through financial, social or educational support.

I am deeply grateful to my supervisor Mohammad Sadegh TALEBI for setting up this internship and the insightful and precious discussion over whiteboard we had. Special thanks to the members of the DeLTA group, Yunlian, Chloé, Yi Shan, Saeed, and particularly Hippolyte for helping me furnish my room upon arrival and Yije for all the tennis and the yellow jersey.
\end{acknowledgements}

%----------------------------------------------------------------------------------------
%	LIST OF CONTENTS/FIGURES/TABLES PAGES
%----------------------------------------------------------------------------------------

\tableofcontents % Prints the main table of contents

%\listoffigures % Prints the list of figures

%\listoftables % Prints the list of tables

%----------------------------------------------------------------------------------------
%	ABBREVIATIONS
%----------------------------------------------------------------------------------------

%\begin{abbreviations}{ll} % Include a list of abbreviations (a table of two columns)

%\textbf{LAH} & \textbf{L}ist \textbf{A}bbreviations \textbf{H}ere\\
%\textbf{WSF} & \textbf{W}hat (it) \textbf{S}tands \textbf{F}or\\

%\end{abbreviations}

%----------------------------------------------------------------------------------------
%	PHYSICAL CONSTANTS/OTHER DEFINITIONS
%----------------------------------------------------------------------------------------

%\begin{constants}{lr@{${}={}$}l} % The list of physical constants is a three column table

% The \SI{}{} command is provided by the siunitx package, see its documentation for instructions on how to use it

%Speed of Light & $c_{0}$ & \SI{2.99792458e8}{\meter\per\second} (exact)\\
%Constant Name & $Symbol$ & $Constant Value$ with units\\

%\end{constants}

%----------------------------------------------------------------------------------------
%	SYMBOLS
%----------------------------------------------------------------------------------------

%\begin{symbols}{lll} % Include a list of Symbols (a three column table)

%$a$ & distance & \si{\meter} \\
%$P$ & power & \si{\watt} (\si{\joule\per\second}) \\
%Symbol & Name & Unit \\

%\addlinespace % Gap to separate the Roman symbols from the Greek

%$\omega$ & angular frequency & \si{\radian} \\

%\end{symbols}

%----------------------------------------------------------------------------------------
%	DEDICATION
%----------------------------------------------------------------------------------------

%\dedicatory{For/Dedicated to/To my\ldots} 

%----------------------------------------------------------------------------------------
%	THESIS CONTENT - CHAPTERS
%----------------------------------------------------------------------------------------

\mainmatter % Begin numeric (1,2,3...) page numbering

\pagestyle{thesis} % Return the page headers back to the "thesis" style

% Include the chapters of the thesis as separate files from the Chapters folder
% Uncomment the lines as you write the chapters

% Chapter 1

\chapter{Introduction}
\label{chapter1}

%------------------------------------------------------------------------------

% Command definition

%------------------------------------------------------------------------------

\section{Motivation}

Reinforcement Learning \citep{sutton_reinforcement_2018} is an active area of machine learning research, that deals with online control problems: how should an agent behave depending on an unknown environment. Most algorithms adapt methods from Dynamic Programming, using e.g. function approximation to try and learn from trajectories of transitions; and overall get good results. However, as for most of the machine learning problems, RL methods ususally suffer a lot from a growing size of the environment. To handle such difficulty, the paradigm of Hierarchical Reinforcement Learning was introduced, with the aim of exploiting apparent structure in the problem to alleviate the quantity of information needed to learn. Temporal abstraction \citep{sutton_between_1999} is probably the main strategy, where decisions aren't made at every steps; it has received a lot of attention lately, and produced good results \citep{fruit_exploration--exploitation_2017, bacon_option-critic_2016, machado_temporal_2021}. Yet, while many speculations have been formulated and experiments conducted \citep{nachum_why_2019,jong_utility_2008}, it is still not clear why, how and when temporal abstraction guarantees improvement; in other words, theoretical understanding of the difference from classical RL is still lacking in the literature. Thus, this thesis aims at deepening it through the study of an important algorithm.


\section{Outline}

Basic theory of Markov Chains and Concentration Inequalities are first presented in Chapter \ref{chapter2}, as those are crucial theoretical tools for the performance analysis of RL algorithms. In the same chapter will also be introduced Markov Decision Processes, the mathematical model at the heart of RL, before describing quickly the horizon of Reinforcement Learning.

Then a quick state-of-the-art of temporal abstraction in RL will be done in Chapter \ref{Chapter3}, so that the reader gets an idea of the main approaches and difficulties of the field. That will then finish to motivate the work in this thesis.

Finally, the theoretical proof for the sample complexity of one of the first algorithms for temporally-extended actions is displayed in Chapter \ref{Chapter4}.

% Chapter 2

\chapter{Sequential Decision Making under Uncertainty}

\label{chapter2}

%------------------------------------------------------------------------------

% Command definition

%------------------------------------------------------------------------------

\section{Markov Chains \& Concentration Inequalities}

Statistical analysis in real-world processes is often applied to a sequence of random variables, for instance to queues of customers in an airport: given an arrival and a service rates, how many customers do I have in queue at any time? The field of concentration inequalities in particular provides probability tail bounds on the distance from a function of such sequence to its expectation; the most studied case is when the random variables are independent, for which the literature offers many sharp bounds. However, independency is a strong assumption, that isn't always relevant; hence the need to consider other settings. Such are Markov Chains, which model the generation of a sequence of states under what is called the Markov property. It also is a stepping stone towards Markov Decision Processes.

\subsection{Markov Chains}
\label{subsec:MC}

A \emph{Markov Chain} on a countable state space $\mathcal{S}$ is a discrete time stochastic process $X_0, X_1, X_2, \dots \in \mathcal{S}$ satisfying the Markov property. In other words, a Markov Chain moves from one state to another following a transition kernel that is without memory, i.e. only depends on the current state and not on the history before.

\begin{defi}[Markov Property]
  $$\forall (n, s) \in \mathbb{N} \times \mathcal{S},\quad \mathbb{P}(X_{n+1} = s | X_0, \dots, X_n) = \mathbb{P}(X_{n+1} = s | X_n)$$
\end{defi}

A Markov Chain is \textbf{time homogeneous} if the transition probability between two states does not depend on the time $n$. In which case, we have:

\begin{defi}[Transition matrix/kernel] $P: \mathcal{S}^2 \to [0, 1]$ is the transition kernel of $(X_n)_n$ if:
  $$\forall (i, j) \in \mathcal{S}^2,\, \forall n \in \mathbb{N},\quad \mathbb{P}(X_{n+1} = j | X_n = j) = P(i, j)$$
  And if $\mathcal{S}$ is finite, $P = (p_{i, j})_{i, j}$ is the $S \times S$ transition matrix of the Markov Chain.
\end{defi}

Then, one can see that $p_{i, j}^m = \mathbb{P}(X_m = j \,|\, X_0 = i)$, the \emph{$m$-step transition probability}, is the element of the matrix $P^m$ at index $(i, j)$, for all $(i, j, m) \in \mathcal{S}\times \mathcal{S} \times \mathcal{N}$.

Now that the basic concepts have been introduced, let's define some properties of states and Markov Chain, in order to better describe the dynamics behind.

Two states $s, s'$ are said to \emph{communicate} if there exists $m, n \in \mathcal{N}^2$ such that $p_{s, s'}^m > 0$ and $p_{s', s}^n > 0$, i.e. there's a path of non-null probability from $s$ to $s'$ and from $s'$ to $s$. Hence, since the ``communicate'' relation is an equivalence relation, $\mathcal{S}$ can be divided into communication classes. In particular, a Markov Chain is said to be \emph{irreducible} if there is only one communication class, in other words: there exists a path of non-null probability between every pair of states.

To describe how close two states are in the Markov Chain, we define the \emph{hitting time}:
\begin{defi}[Hitting time]
  $$\forall \, s \in \mathcal{S},\quad T^s = \inf \{n \in \mathcal{N}^* \,|\, X_n = i\}$$
  the time of first visit to a state by the Markov Chain.
\end{defi}

Then, depending on the behaviour of the hitting time, a state is said to be \emph{transient} or \emph{recurrent}:
\begin{defi}[Recurrence]
  A state $s \in \mathcal{S}$ is \textbf{recurrent} if and only if
  $$\mathbb{P}(T^s < +\infty \,|\, X_0 = s) = 1$$ otherwise it is \textbf{transient}.\\
  Moreover, if $s$ is recurrent and
  $$\mathbb{E}[T^s \,|\, X_0 = s] < +\infty$$
  then $s$ is said to be \textbf{positive recurrent}, otherwise it is \textbf{null recurrent}.
\end{defi}

In the finite state space case, any recurrent state is positive recurrent. Also, it is interesting to observe that all states in a communication class are either recurrent or transient.

%Furthermore, a state $s$ is said to be \emph{periodic} with period $k \in \mathcal{N}$ if all $m \in \mathcal{N}^*$ such that $p_{s, s}^m > 0$ are divisible by $k$ and $k > 1$. If $k = 1$, then $s$ is \emph{aperiodic}. Again, it is a property shared among a same class.

Finally, we can define \emph{ergodicity}:
\begin{defi}[Ergodicity]\label{ergodicity}
  %A state is ergodic if it is positive recurrent and aperiodic.\\
  %A Markov Chain is ergodic if all states are.\\
  A Markov Chain is ergodic if it is irreducible and positive recurrent.
\end{defi}

Intuitively, in an ergodic Markov Chain all states will be visited regularly and without a periodic pattern. This behaviour is described by the following theorem:
\begin{thm}[Ergodic Theorem]
  Let $P$ be the transition matrix of an ergodic Markov Chain. Then there exists a unique probability distribution $\pi$ over the state space that solves the system of equations $\pi^T = \pi^T P$. $\pi$ is called the \textbf{stationary distribution} of $P$.
\end{thm}

The theorem above motivates the definition of ergodicity: it says that, as time tends towards infinity, the Markov Chain is doomed to fall into steady dynamics over all the state space. Its stationary distribution gives the probability to be in a certain state at an unknown time, after a burn-in period. In particular, this behaviour is independent of the initial distribution. For these reasons, ergodicity is a common property to assume when analysing the theoretical guarantees of an algorithm.

\subsection{Concentration inequalities}
\label{subsec:concentration}

The goal of concentration inequalities is to provide bounds on tail probabilities, in the form $\mathbb{P}(|Z - \mathbb{E}[Z]| > t) \leq C_Z$. It is particularly useful when analysing algorithms to control the likelihood of extreme events to occur, and then to provide a general bound with high probability.

\subsubsection{Generalised Markov Inequality}

The Markov inequality is an important inequality linking tail probabilities of a random variable to its expectation.

\begin{thm}[Generalised Markov inequality]
  Let $Y \geq 0$ be a random variable, and $\phi$ be a positive and non-decreasing function.
  $$\forall \, t > 0,\quad \mathbb{P}(Y \geq t) \leq \mathbb{P}(\phi(Y) \geq \phi(t)) \leq \frac{\mathbb{E}[\phi(Y)]}{\phi(t)}$$
  In particular: $\mathbb{P}(Y \geq t) \leq \frac{\mathbb{E}[Y]}{t}$
\end{thm}

\begin{proof}
  $$(Y \geq t) \implies (\phi(Y) \geq \phi(t))$$ so $\mathbb{P}(Y \geq t) \leq \mathbb{P}(\phi(Y) \geq \phi(t))$. Moreover, $$\phi(t) \mathbb{1}_{\phi(Y) \geq \phi(t)} \leq \phi(Y)$$ hence by taking the expectation: $$\phi(t) \mathbb{P}(\phi(Y) \geq \phi(t)) \leq \mathbb{E}[\phi(Y)]$$ (here is used the well-known formula $\mathbb{E}[\mathbb{1}_{A}] = \mathbb{P}(A)$ for any event $A$).
\end{proof}

This simple inequality is actually quite useful, as displayed below.

\begin{thm}[Chebychev's inequality]
  Let $Y$ be a real random variable such that $\mathbb{E}[Y] < +\infty$, $\mathbb{V}[Y] = \mathbb{E}[(Y - \mathbb{E}[Y])^2] < +\infty$.
  $$\forall \, t > 0,\quad \mathbb{P}(|Y - \mathbb{E}[Y]| \geq t) \leq \frac{\mathbb{V}[Y]}{t^2}$$
\end{thm}

But most importantly, Markov's inequality is used in the Cramer-Chernoff method, presented below.

\subsubsection{Cramér-Chernoff method}

The Cramér-Chernoff is a general inequality with exponential decay that can be used to derive interesting results, as we will see in the next subsection. It is based on a transformation called the \emph{cumulant generative function}:

\begin{defi}[Cumulant generative function]
  Let $Y$ be a real random variable. We define the following quantities:
  \begin{enumerate}
  \item Where it exists, the \textbf{moment-generating function}: $M_Y(\lambda) = \mathbb{E}[e^{\lambda Y}]$
  \item On the same set, the \textbf{cumulant generative function}: $\psi_Y(\lambda) = \log M_Y(\lambda) = \log( \mathbb{E}[e^{\lambda Y}])$
  \item And accordingly, the \textbf{Cramér-Chernoff transform}: $\forall \, t \in \mathbb{R},\quad \psi_Y^*(t) = \sup_{\lambda \geq 0} \lambda t - \psi_Y(\lambda)$
  \end{enumerate}
\end{defi}

If the tail of a random variable doesn't decay fast enough (i.e. exponentially), the exponential function in the definition of the moment-generating function will make it grow too much. This is why a random variable $Z$ is qualified \emph{heavy-tailed} if $M_Z(\lambda) = + \infty$ for all $\lambda > 0$; otherwise it is \emph{light-tailed}.

With these quantities in mind:
\begin{thm}[Chernoff's inequality]
  $$\forall \, t \in \mathbb{R},\quad \mathbb{P}(Y \geq t) \leq e^{- \psi_Y^*(t)}$$
\end{thm}

\begin{proof}
  From the Generalised Markov inequality, $\mathbb{P}(Y \geq t) \leq e^{-\lambda t} \mathbb{E}[e^{\lambda Y}]$ for all $\lambda \geq 0$. $\psi_Y^*(t)$ is a minimizer of this upper bound, hence the result.
\end{proof}

This inequality is the main ingredient to many important concentration inequalities, like Bernstein's or Azuma's. The approach consists in applying Chernoff's inequality to more specific cases, thus playing with the moment-generating function.

\subsubsection{Hoeffding's inequality}

The following result is very valuable, and rely on a specific class of random variables:

\begin{defi}[$\sigma^2$-Subgaussian variable]
  Let $Y$ be a real random variable. The following relations are equivalent:
  \begin{enumerate}
  \item $\exists \sigma > 0,\ \forall \, \lambda \in \mathbb{R},\quad \mathbb{E}[\psi_Y(\lambda)] = \mathbb{E}[e^{\lambda Y}] \leq e^{\lambda^2 \sigma^2 / 2}$
  \item $\exists \sigma > 0,\ \forall \, \lambda > 0,\quad \mathbb{P}\left(|Y| \geq \lambda \right) \leq 2e^{-\frac{\lambda^2}{2\sigma^2}}$
  \end{enumerate}
  In which case $Y$ is said to be $\sigma^2$-subGaussian.
\end{defi}

Those inequalities become equalities if $Y$ is Gaussian with variance $\sigma^2$. It means that the tail probabilities of $Y$ are decreasing at least as fast as for a $\sigma^2$ Gaussian distribution.

An interesting property of subGaussianity is the following:
\begin{thm}
  If $(X_i)_{1 \leq i \leq n}$ i.i.d. and $\sigma_i^2$-subGaussians, then $\sum_{i=1}^n X_i$ is $(\sum_{i=1}^n \sigma_i^2)$-subGaussian, and $aX_1 + b$ is $(a^2\sigma_1^2)$-subGaussian
\end{thm}

Although the subGaussian property might look a bit difficult to showcase, the next lemma provides it for very reasonable assumptions:

\begin{lem}[Hoeffding's lemma]
  Let $Y$ be a centered real random variable. If $Y \in [a, b]$, then $\mathbb{V}[Y] \leq \frac{(b - a)^2}{4}$ and $Y$ is $\frac{(b-a)^2}{4}$-subGaussian.
\end{lem}

Bounded random variables are subGaussians, so how can we use that?

\begin{thm}[Hoeffding's inequality]\label{thm:hoeffding}
  Let $X_1, \dots, X_n$ be independent random variables such that $X_i \in [a_i, b_i]$ almost surely. Then, for any $t \in \mathbb{R}$:
  $$\mathbb{P}\left(\frac{1}{n} \sum_{i=1}^n (X_i - \mathbb{E}[X_i]\right) \geq t) \leq e^{-\frac{2 t^2 n^2}{\sum_{i=1}^n (b_i - a_i^2}}$$
\end{thm}

As one can see, this result provides an interesting bound for rather mild assumptions. The derivation of this inequality consists in applying Hoeffding's lemma to prove the sum of variables is subGaussian, then use the corresponding probability bound.

\section{Markov Decision Processes}
\label{section:MDP}

After introducing Markov Chain, we will now move on to the next level with \emph{Markov Decision Processes}, which add a layer of control. This mathematical object is at the heart of Reinforcement Learning, by modeling a vast variety of situations where decision making takes place, such as chess, autonomnous driving or robotics.

\subsection{Definitions}
\label{subsec:MDP-defs}

\begin{defi}[Markov Decision Process]
  A Markov Decision Process (MDP) $\mathcal{M}$ is a tuple $\langle \mathcal{S}, \mathcal{A}, p, r \rangle$ where:
  \begin{enumerate}
  \item $\mathcal{S}$ is the \textbf{state space}
  \item $\mathcal{A}$ is the \textbf{action space}
  \item $p $ is the \textbf{transition probability kernel} such that $p(s' \,|\, s, a)$ is the probability to move to state $s'$ when taking action $a$ from state $s$. As for Markov Chains, it has the Markov property.
  \item $r: \mathcal{S} \times \mathcal{A} \to [0,1]$ is the \textbf{reward function}, bounded by $0$ and $1$ for convenience and without loss of generality.
  \end{enumerate}
  $\mathcal{M}$ is \textbf{finite} if $\mathcal{S}$ and $\mathcal{A}$ are, in which case we denote $S = |\mathcal{S}|$, $A = |\mathcal{A}|$.\\
  In general the reward is stochastic (i.e. drawn from a probability distribution specific at each state-action pair), however in practice only the expected reward matters, so we already assume a deterministic reward function for convenience.
\end{defi}

In decision making problems, an \emph{agent} moves stochastically from state to state in an MDP by taking actions, and depending on its trajectory and decisions receives rewards along the way. This generates a \textbf{history} of transitions $(s_i, a_i, r_i)_{i \in \mathcal{N}}$. Hence the goal is to learn a (near-)optimal decision rule to maximise the rewards, from such history/ies. The agent starts from an initial state based on an \emph{initial distribution} $\mu_0 \in \mathbf{\Delta}(\mathcal{S})$.

A decision rule is defined as a \emph{policy} $\pi : \mathcal{S} \to \Delta(\mathcal{A})$, where $\Delta(\mathcal{A})$ is the set of probability measures over $\mathcal{A}$. $\pi$ is \emph{deterministic} if it maps each state to only one action: $\pi : \mathcal{S} \to \mathcal{A}$. And a policy is \emph{stationary} if it does not depend on the current time step.

Following one stationary policy on $\mathcal{M}$ induces a Markov Reward Process, which is a Markov Chain with rewards upon transitions. In this case, the transition probability kernel follows $\mathbb{P}^\pi(s' \,|\, s) = \mathbb{P}(s' \,|\, s, \pi(s))$.

This link with Markov Chains leads to the following definitions:

% CITE RONAN'S THESIS
\begin{defi}[Classification of MDPs]
  An MDP is:
  \begin{enumerate}
  \item \textbf{Ergodic} if the Markov Chain induced by any deterministic stationary policy is ergodic
  \item \textbf{Unichain} if the Markov Chain induced by any deterministic stationary policy is composed of one ergodic sub-Markov Chain in addition to a set of transient states
  \item \textbf{Communicating} if there exists a deterministic stationary policy such that the induced Markov Chain is communicating
  \end{enumerate}
\end{defi}

The ergodic property ensure that, whatever the policy used during learning, all states will be visited infinitely often; unichain means ergodicity except for some states that will not be visited anymore as times goes; and communicating ensure the existence of a path between every pair of states.

\subsection{Settings}
\label{subsec:MDP-settings}

As said before, the goal is to learn to make the best decisions, in other words to find a (near-)optimal policy; however different learning settings exist and are considered in the literature.

A control problem over a MDP is \emph{finite horizon} if only the $H$ first steps are considered, with $H \in \mathbb{N}^*$. Otherwise it is \emph{infinite horizon}. This difference is obviously significative, as the value of a state might change if future rewards can't be obtained because of a finite horizon. The initial state distribution is particularly important in this case, since a high distance from the initial state can make a profitable state not worth it/possible to get to.

A variant of those settings is the \emph{episodic} problem: the agent plays forever, but every $H$ steps it is teleported back to an initial state accordingly to $\mu_0$. It is different from finite horizon since the agent plays this game not just once but infinitely many times, and it is different from infinite horizon as the agent is ``tied'' to those initial states.

Another parameter of such problems is the \emph{discount factor} $\gamma \in [0, 1]$. This is a multiplicative discount applied to rewards based on the time at which they are received, i.e. the reward $r_t$ is actually perceived as $\gamma^t r_t$ by the agent. The motivation behind a discount factor is mathematical, displayed in the next subsection, but intuitively the idea is to penalise long-term reward because of the uncertainty of getting them, while short-term rewards are more reliable. The choice of $\gamma$ is crucial as it is involved in the complexity bounds of many algorithms; as a rule of thumb, we will see that in the discounted setting, $1/(1 - \gamma)$ can be seen as the \emph{effective horizon} of the problem.

\subsection{Measures of performance}
\label{subsec:MDP-measures}

As there are different possible learning settings, there are different possible learning objectives.

The most common one is maximising the \emph{cumulative reward}:

\begin{defi}[Cumulative reward]
  Given a sequence of transitions $(s_t, a_t, r_t)_t$ from an agent in a MDP, we define the cumulative reward $R$ as being:
  \begin{enumerate}
  \item In the finite horizon setting: $\sum_{t=0}^{H}r_t$
  \item In the infinite horizon discounted setting: $\sum_{t=0}^{+\infty} \gamma^t r_t$
  \item In the infinite horizon undiscounted setting: $\liminf_{T \to \+\infty} \frac{1}{T}\sum_{t=1}^{T} r_t$
  \end{enumerate}
  The last definition is more known as the \emph{gain/long-term average reward}, and requires the use of a $\liminf$ to ensure good definition. It represents the asymptotic per-step average reward.\\
  In the infinite horizon undiscounted setting, one can easily see that, since $0 \leq r \leq 1$, $R \leq \frac{1}{1 - \gamma}$; this justifies the use of a discount factor strictly less than one for infinite horizon, otherwise the cumulative reward might diverge.
\end{defi}

When analysing algorithms performances, theoretical results often focus on the \emph{expected cumulative reward}, which is simply the expectation of the cumulative reward.\\[1.5\lineskip]

From now on, \textbf{we shall focus on the finite case infinite horizon discounted setting}.\\[1.5\lineskip]

A more insightful measure is the \emph{regret}: $$\sum_{t=0}^{+\infty} \gamma^t(r_t - r_t^*)$$where $r_t^*$ is the optimal expected reward at time $t$ (the expected reward when taking the optimal action in state $s_t$).

However, while those quantities focus on the ability of the agent to get a lot rewards, one might be more interested in minimising the time needed to learn an $\epsilon$-optimal policy. The $(\epsilon, \delta)$ sample complexity of an agent is the number of time steps $T$ needed before $\| \mathbf{V}_t^\pi - \mathbf{V}^* \|_\infty \leq \epsilon$ for all $t \geq T$ with probability $1 - \delta$, where $\mathbf{V}_t^\pi$ is the value of the policy of the agent at step $t$ and $\mathbf{V}^*$ is the optimal value (values are defined in the next section). Intuitively, this is the PAC convergence speed to a policy.

Finally, another version of sample complexity is the (asymptotic) upper-bound on the number of steps where the current policy isn't $\epsilon$-optimal. In this case, the emphasis is put on minimising the number of mistakes rather than on the learning speed.

Now we shall introduce some crucial notion with respect to those measures of learning performance.

\subsection{Values \& Optimality}
\label{subsec:MDP-values}

In order to get to optimality, many MPD algorithms assign values to states or state-action pairs. These quantities guide decision making towards the most profitable choice, and allow to progressively improve the value of a policy.

\begin{defi}[Value \& Q-value]
  For any policy $\pi$, we define for all state-action pair $(s, a) \in \mathcal{S}\times \mathcal{A}$:
  \begin{align*}
  V^\pi(s) &= \mathbb{E}_\pi \left[\sum_{t=0}^{+\infty} \gamma^t r(s_t, a_t) \,|\, s_0=s \right]\\
  Q(s, a)^\pi &= \mathbb{E}_\pi \left[\sum_{t=0}^{+\infty} \gamma^t r(s_t, a_t) \,|\, s_0=s, a_0=a \right]
  \end{align*}
  From these definitions, one can see that $V^\pi(s) = Q(s, \pi(s))$.
  The \textbf{value} and the \textbf{Q-value} functions of $\pi$, respectively. The expectation $\mathbb{E}_\pi[\dots]$ means that the actions $a_t$ are taking accordingly to $\pi(s_t)$.\\
  In the finite case, we denote $\mathbf{V}^\pi$ and $\mathbf{Q}^\pi$ the vectors of values and Q-values of $\pi$.
\end{defi}

As one can see, those quantities represent what an agent can expect to get as rewards in the future, from its current state/state-action. The value can guide the decision making, as with the \emph{greedy policy} $\pi_g(s) = \arg\max_{a \in \mathcal{A}} Q(s, a)$. From here, it is easy to define what an \emph{optimal} policy is:

\begin{defi}[Optimality]
  A policy $\pi^*$ is optimal if $$\forall \, s \in \mathcal{S},\quad \pi^* \in \arg\max_{\pi \in \Pi} V^\pi(s)$$ where $\Pi$ is the set of all policies.\\
  We then denote $V^{\pi^*} = V^*$, the optimal state value function.\\
  Similarly, $Q^{\pi^*} = Q^*$ and $V^*(s) = \max_{a \in \mathcal{A}} Q^*(s, a)$ for all $s \in \mathcal{S}$.\\
  Moreover, there exists one optimal policy $\pi^*_d$ that is deterministic, and $\pi^*_d(s) = \arg\max_{a \in \mathcal{A}} Q^*(s,a)$ for all $s \in \mathcal{S}$.
\end{defi}

One can already see how relevant the value of a policy is when seeking optimality, but what makes those functions so interesting is the following property:

\begin{defi}[Bellman operator]
  The Bellman operator associated to a policy $\pi$ is $$\mathcal{T}^\pi : \begin{cases} &\mathbb{R}^S \to \mathbb{R}^S\\ V &\mapsto \mathcal{T}^\pi(V) \end{cases}$$ where $$\forall \, s \in \mathcal{S},\quad \mathcal{T}^\pi(V)(s) = r(s, \pi(s)) + \gamma \sum_{s'\in \mathcal{S}} p(s' \,|\, s, \pi(s)) V(s')$$
  $\mathcal{T}^\pi$ being a $\gamma$-contraction, $V^\pi$ is the only fixed point: $$\forall \, s \in \mathcal{S},\quad V^\pi(s) = r(s, \pi(s)) + \gamma \sum_{s'\in \mathcal{S}} p(s' \,|\, s, \pi(s)) V^\pi(s')$$\\
  Thanks to this result, the greedy policy can be obtained from $V$: $$\pi_g(s) = \arg\max_{a \in \mathcal{A}} \left[ r(s,a) + \gamma \sum_{s' \in \mathcal{S}}p(s' \,|\, s, a) V^*(s') \right]$$
  A similar result holds for the Q-value: $$\forall \, (s, a) \in \mathcal{S}\times \mathcal{A},\quad Q^\pi(s, a) = \mathcal{T}^\pi(Q)(s,a) = r(s, a) + \gamma \sum_{s' \in \mathcal{S}} p(s \,|\, s,a) V^\pi (s')$$
  When considering any optimal policy, one get the \textbf{optimal Bellman operator}: $$\mathcal{T}^*(V)(s) = \max_{a \in \mathcal{A}} \left[ r(s, a) + \gamma \sum_{s' \in \mathcal{S}} p(s' \,|\, s,a)V(s') \right]$$
  Since $\mathcal{T}^*$ is a $\gamma$-contraction as well, iteratively applying $\mathcal{T}^*$ over an initial value function $V_0$ converges towards $V^*$.
\end{defi}

As the reader can realise from the above properties of the Bellman operator, we just acquired a great theoretical tool on our way towards optimality. The significance of the Bellman operator is showcased in the following section.


\section{Reinforcement Learning basics}
\label{section:RL}

In this section, we shall present the basic approaches to solve a MDP problem. While those algorithms have been knwon for some time already, they still represent the foundation of many of the latest ones, and are still actively studied.

\subsection{Planning}
\label{subsec:RL-planning}

Actually preceding Reinforcement Learning is the field of \emph{planning}: planning is the problem of seeking optimality when the MDP is fully known, specifically $p$ and $r$. The results introduced in the previous section allow us to directly present \emph{Value Iteration} (algorithm \ref{alg:VI}).

\begin{algorithm}[htbp]
\small
\caption{Value Iteration}
\label{alg:VI}
\begin{algorithmic}
\State \textbf{Initialization:} $\mathbf{V_0} \in \mathbb{R}^S$ initial value, $\epsilon > 0$ error margin.
\State $\mathbf{V} \leftarrow \mathbf{V_0}$
\While{$\| \mathbf{V} - \mathcal{T}^*(\mathbf{V}) \|_\infty \geq \epsilon$}
\State $\mathbf{V} \leftarrow \mathcal{T}^*(\mathbf{V})$
\EndWhile\State
\Return $\pi_g$ the greedy algorithm with respect to $\mathbf{V}$.
\end{algorithmic}
\normalsize
\end{algorithm}

In this approach, the optimisation is done in the world of the value functions, and an optimal policy is obtained with the greedy operator. The next algorithm though, \emph{Policy Iteration} (algorithm \ref{alg:PI}), operates directly on the world of policies via evaluation and then improvement from the evaluation.

\begin{algorithm}[htbp]
\small
\caption{Policy Iteration}
\label{alg:PI}
\begin{algorithmic}
\State \textbf{Initialization:} $\pi_0 \in \Pi$ intial policy.
\State $\pi_1 \leftarrow \pi_0$
\State $\pi_2 \leftarrow \emptyset$
\While{$\pi_1 \neq \pi_2$}
\State $\pi_2 \leftarrow \pi_1$
\State Compute/estimate $V^{\pi_1}$ \Comment{Policy evalutation}
\State $\pi_1 \leftarrow \mathsf{greedy}(V^{\pi_1})$ \Comment{Policy improvement}
\EndWhile\State
\Return $\pi_1$
\end{algorithmic}
\normalsize
\end{algorithm}


It is interesting to notice how the Policy Iteration requires a policy evaluation algorithm (such as Value Iteration); this highlight the power of value functions.

\subsection{Various Reinforcement Learning paradigms}
\label{subsec:RL-paradigms}

However in real-world application, one is more often interested in Reinforcement Learning problems, where \emph{$r$ and $p$ are unknown} ($\mathcal{S}$ and $\mathcal{A}$ can usually be assumed to be known). Because of this lack of knowledge, the previous algorithms can't be used right away. To handle this difficulty, many strategies have been developped.

First of all, we can distinguish \emph{model-free} and \emph{model-based} algorithms: in the latter, estimates of $p$ and $r$ are first computed, and then used in a planning algorithm; while a model-free approach directly estimates values or policies.

To build those estimates, multiple techniques are possible, such as Monte-Carlo or Temporal Difference. However, Monte-Carlo methods require a significant number of simulations that grows with $S$ and $A$ to work, while on the contrary Temporal Difference updates use very little data, thus possibly missing some crucial information on the MDP. In that respect, an interesting notion is the \emph{Optimism in the face of Uncertainty}: adding a bias in estimates that favors least explored states. This illustrates the \emph{Exploration-Exploitation} trade-off problem in online learning: exploration is costly but necessary to take good decisions. Another example is the use of the \emph{$\epsilon$-greedy} policy, where with probability $\epsilon$ a random action is taken, otherwise the greedy action is followed.

Furthermore, a RL algorithm is said to be either \emph{on-policy} or \emph{off-policy}: in the first case, the policy used to simulate and collect samples is the policy improved and returned, while in the second case a \emph{behavior policy} is followed during execution while the data is used to improve the target policy.

These ideas are showcased in the enxt subsection.

\subsection{Q-Learning}
\label{subsec:Rl-QL}

Q-Learning (QL) is one of the oldest and most important algorithm in RL. Indeed, numerous variants have been proposed since, and proofs for theoretical guarantees are still being published nowadays.

QL is a model-free off-policy that uses Temporal Difference updates to maintain an estimate of $Q^*$, which then can be used to get $\pi^*$. It is very simple, as shown in algorithm \ref{alg:QL}.

\begin{algorithm}[htbp]
\small
\caption{Asynchronous Q-Learning}
\label{alg:QL}
\begin{algorithmic}
\State \textbf{Initialization:} $\pi_b \in \Pi$ behaviour policy, $(\alpha_i)_{i \in \mathbb{N}}$ learning rates, $Q_0 \in \mathbb{R}^{S \times A}$ initial Q-value matrix, $s_0$ initial state, $N_0(s,a) = 0 \ \forall (s,a)$ the number of visits of $(s,a)$ so far.
\For{$t=1,2,\ldots$}
\State Sample action $a_t\sim \pi_b(s_t)$ and observe $s_{t+1}\sim P(\cdot\vert s_t,a_t)$.
\State $N_{t+1}(s_t,a_t)=N_t(s_t,a_t)+1$
\State $\alpha = \alpha_{N_t(s_t,a_t)+1}$
\State $Q_{t+1}(s_t,a_t)= (1-\alpha)Q_t(s_t,a_t)+\alpha\Big(r(s_t,a_t)+\gamma \max_{a\in\mathcal{A}}Q_t(s_{t+1},a)\Big)$.
\EndFor
\end{algorithmic}
\normalsize
\end{algorithm}

In synchronous QL, the behaviour policy is replaced by $\mathbf{greedy}(Q_t)$, or $\epsilon-\mathbf{greedy}(Q_t)$ to keep on exploring. Also, many tweaks can be made such as using optimistic initial values or maintaining two estimates of $Q$. A common choice for the learning rate is $\alpha_i = \frac{1}{i}$.

The core idea of Temporal Difference is that $r(s_t,a_t)+\gamma \max_{a\in\mathcal{A}}Q_t(s_{t+1},a) = \mathcal{T}^*(Q)(s,a)$ is actually an estimate of $Q^*$.

\begin{thm}[Convergence of Q-Learning]\label{thm:QL-conv}
  If all state-action pairs are visited infinitely often and $$\sum_{i=1}^{+\infty}\alpha_i = +\infty \quad\text{and}\quad \sum_{i=1}^{+\infty}\alpha_i^2 < +\infty$$ then $Q_t$ will converge to $Q^*$ with probability 1 as $t$ tends towards infinity.
\end{thm}

This result justify the QL algorithm. However, it is very well known that, even if the QL update is very cheap to execute, the convergence can be very slow, hence QL is not \emph{sample efficient}.

Please note that for all state-action pairs to be visited infinitely often is equivalent to say that the Markov Chain induced by $\pi_b$ is ergodic.
 
% Chapter Template

\chapter{Hierarchical Reinforcement Learning by Temporal Abstraction} % Main chapter title
\label{Chapter3}

So far, a very wide category of MDPs has been considered, with very few assumptions on the state-action space. However, in many cases the agent could make use of some regular behaviour in the MDP to learn faster/more from each experience. This is the domain of Hierarchical Reinforcement Learning, in which the MDP is decomposed into subproblems, thus creating different scales of learning. A particular instance of such approach is Temporal Abstraction: exploiting temporal regularities to reduce the problem complexity. This is a step towards human thinking, as it corresponds to learning to look at the big picture, rather than keeping eyes on our feet.

Multiple frameworks have been suggested in the literature, but we will introduce the most adopted one in the next section.

\section{Options framework and Semi-Markov Decision Processes}

\subsection{Options}
\label{subsec:options}

The \emph{option framework} introduced in \citep{sutton_between_1999} formalizes the notion of macro-actions: viewing a sequence of actions as a larger action towards a goal.

\begin{defi}[Options]
  An option is a triple $\langle \mathcal{I}, \pi, \beta \rangle$, where:
  \begin{enumerate}
  \item $\mathcal{I} \subseteq \mathcal{S}$ is the \textbf{initiation set}
  \item $\pi \in \Pi$ is the option policy
  \item $\beta: \mathcal{S} \to [0, 1]$ is the termination condition
  \end{enumerate}
  The option can be started from any state $s \in \mathcal{I}$, meaning the agent follows policy $\pi$ until it terminates at a state $s'$ with probability $\beta(s')$.\\
  An action $a \in \mathcal{A}$ defines a \textbf{primitive option} $o_a$ which can start everywhere, always takes action $a$ and terminates with probability 1 in every state.
\end{defi}

As one can see, choosing an option amounts to following a (winning?) strategy depending on the current location, until another cross-road state is reached, thus demanding for a new decision. In that respect, an agent can start acting with a well defined set of options rather than actions. The direct benefit is to potentially largely reduce the number of state-actions pairs. This is temporal abstraction in the sense that the there's a varying number of steps between two decisions.

\begin{defi}[Options Framework]
  A set of options $\mathcal{O}$ is \textbf{well-defined} if there is always an option available for the agent to play. More specifically, the union of initiation sets includes initial states and terminal states (states with non-null probability that an option temrinates at). An option set including all the primitive actions is obviously well-defined.\\
  In such case, one can define a \textbf{policy over options} $\mu: \mathcal{S}\times \mathcal{O} \to [0,1]$, that an agent follows at decision epochs. 
\end{defi}

It is now clear that with a well-defined option framwork, an agent can actually deal with the state-option space $\mathcal{S}\times \mathcal{O}$ as a regular state-action space. The following result showcases this similarity:

\begin{thm}[Option values]
  For a policy over options $\mu$, its value is $$V^\mu(s) = \mathbb{E}_\mu \left[ \left(\sum_{t=0}^{k-1}\gamma^t r_t \right) + \gamma^k V^\mu(s_{k}) \,|\, \varepsilon(s, \mu, 0) \right]$$
  where $\varepsilon(s, \mu, t)$ is the event that $\mu$ makes a decision $o_t$ from state $s$ at time $t$, and $k$ is the duration of $o_t$. The \textbf{Q-value function} of $\mu$ is defined similarly.\\
  Then \textbf{Q-Learning for options}, with update $$Q(s, o) = (1 - \alpha)Q(s,o) + \alpha \left( r + \gamma^k V(s') \right)$$ converges, under similar conditions as in Theorem \ref{thm:QL-conv}, and $\mathbb{E}_\mu \left[k \right] < +\infty$.
\end{thm}

In the end, by providing a set of well-defined options as prior knowledge, one can significantly reduce the size of the problem and the number of updates, with almost no change in the algorithm. This difference actually lies in the reward discount: because of the stochastic duration time, not only do we need to apply $\gamma^k$, but also the reward $r$ is actually the discounted cumulative reward of the option.

Even though the number of updates per time step is lesser, it appears to be less sample efficient, in the sense that the steps taken inside an option aren't contributing much. In that respect, \citep{sutton_between_1999} also introduces \emph{Intra-Option Q-Learning}: an update is also done for time steps within options, not only for the current option (weigthed by the termination probability) but as well for options that would have taken the same action in the current state. That way, no experience is lost. And finally, there is the possibility to \emph{interrupt} options, whenever an other option can be started with higher value.

The upsides and downsides of temporal abstraction with the option framework are less obvious than it appears at first thought; this is actually an open question, that has already been the subject of several projects in the literature, some of which will be presented in a later section of the chapter.

However the option framework, even if intuitive, isn't the only approach in that regard; in the next section are introduced \emph{Semi-Markov Decision Processes}, a generalisation of MDPs that includes options.

\subsection{Semi-Markov Decision Processes}
\label{subsec:SMDP}

While classical MDPs evolve at the regular unit time step, it is intuitive that in real-world problems some actions take more time to execute than others, and the duration time can vary even for a same action. SMDPs model this behaviour:

\begin{defi}[Semi-Markov Decision Process]
  A Semi-Markov Decision Process is defined as $M = \langle \mathcal{S}, \mathcal{A}, p, r \rangle$, alike MDPs, with the difference that an agent takes $0<\tau$ time to transition from state $s$ to state $s'$, with probability $p(s', \tau \,|\, s,a)$. For practical reasons, we suppose that $\mathbb{E} \left[\tau \right] < +\infty$. We also assume that $0 < \tau_{min} \leq \tau \leq \tau_{max} \leq +\infty$.\\
  A history related to a SMDP is now made of transitions $\left( s_t, a_t, r_t, \tau_t \right)_t$.\\
  The \textbf{total time} at time step $t$ is defined as $\sigma_t = \sum_{i=0}^{t-1} \tau_i$ ($\sigma_0 = 0$).\\
  The \textbf{total cumulative reward} is now $\sum_{t = 0}^{+ \infty} \gamma^{\sigma_t} r_t$.
\end{defi}

Obviously a SMDP with $\tau=1$ is a MDP. Moreover, the condition on $\tau_{min} \neq 0$ is here to ensure the convergence of the total cumulative reward. Indeed, if $\tau_{\min} = 0$ then one can consider the sequence of duration times $\{\tau_t = 1/(t + 1)^2\}_{t=0}^{\infty}$; it is well known that this series converges toward $\pi^2/6$, so that $\sigma_t < \pi^2/6$ for all $t$, hence $$\sum_{t = 0}^{\infty} \gamma^{\sigma_t} r_t \geq \sum_{t=0}^{\infty} \gamma^{\frac{\pi^2}{6}} = + \infty$$

To be almost sure this doesn't happen and for convenience, we assume $\tau_{min} > 0$.

Now that SMDPs are clearly defined, it is easy to see how a MDP equipped with an option framework naturally defines a SMDP, in which $\tau$ is the number of steps taken by the option/action, and the rewards are redefined as $$r_{smdp}(s, o) = \mathbb{E}_{\pi_o} \left[ \sum_{i=0}^{\tau-1}\gamma^i r_{mdp}(s_i, \pi_o(s_i)) \,|\, s_0=s\right]$$ We also require the options to be defined everywhere on the state space $\mathcal{I} = \mathcal{S}$.

As it has been done for options, we shall adapt the values and Q-values definitions to SMDPs:

\begin{thm}[Values in SMDP]\label{thm:values-SMDP}
  The value and Q-value functions are now defined as:
  \begin{align*}
    \forall s \in \mathcal{S}, \qquad &V^{\pi}(s) := \mathbb{E} \left[\sum_{t = 0}^{\infty} \gamma^{\sigma_{t}} r(s_t, a_t) \, \middle| \,  s_0 = s \right]\\
    \forall (s, a) \in \mathcal{S} \times \mathcal{A}, \qquad &Q^{\pi}(s,a) := \mathbb{E} \left[\sum_{t = 0}^{\infty} \gamma^{\sigma_{t}} r(s_t, a_t) \, \middle| \,  s_0 = s,\, a_0 = a \right]
  \end{align*}
  and the corresponding Bellman operators:
  \begin{align*}
    \mathcal{T}^\pi(Q^{\pi})(s, a) &:= r(s, a) + \mathbb{E}_{(s', t) \sim p(\cdot, \cdot | s, a)} \left[ \gamma^t Q^{\pi}(s', \pi(s')) \right] = Q^{\pi}(s, a)\\
    \mathcal{T^*}(Q)(s, a) &:= r(s, a) + \mathbb{E}_{(s', t) \sim p(\cdot, \cdot | s, a)} \left[ \gamma^t \max_{a' \in \mathcal{A}} Q(s', a') \right]\\
    &= r(s, a) + \int_{s' \in \mathcal{S}} \int_0^{+ \infty} \gamma^t V^*(s') p(s', t | s, a) dt ds' \\
    &= r(s, a) + \int_{s' \in \mathcal{S}} V^*(s') \tilde p(s' | s, a) ds'
  \end{align*}
  where $$\tilde p(s', t | s, a) = \mathbb{E}\left[ \gamma^\tau \,|\, s'\right] = \int_0^{+\infty} \gamma^t p(s', t | s, a) dt$$ is the \textbf{expected discounted probability kernel}.
\end{thm}

We highlight the change in the optimal Bellman operator: because the discount factor is time dependent, and the duration time is not necessarily independent to the next state, $\gamma$ is no longer a free factor in front of the probability matrix $P$.

Hence, \emph{Q-Learning for SMDP} is adapted from QL with the update $$Q_t(s_{t-1}, a_{t-1}) = (1 - \alpha_t)Q_{t-1}(s_{t-1}, a_{t-1}) + \alpha_t \left(r(s_{t-1}, a_{t-1}) + \gamma^{\tau_{t-1}} \max_{a\in \mathcal{A}} Q_{t-1} (s_t, a)\right)$$

The algorithm has been presented in \citep{bradtke_reinforcement_1994}, and its convergence proven in \citep{parr_hierarchical_1998} assuming independency between $\tau$ and $s'$, though the proof can easily be adapted for the general formulation.


\section{Algorithms in the literature}

In this section we do an overview of the literature of temporal abstraction, ranging from learning/discovering options to model-based algorithms for SMDPs.

Temporal abstraction has been there for a long time already (\cite{parr_hierarchical_1998, fikes_learning_1972}), presented as a first step towards human intelligence, and a fitting approach to many settings, with promising benefits. Yet, the option framework introduced in \citep{sutton_between_1999} has really set up the reference for temporal abstraction in reinforcement learning for years to come. Indeed, thanks to the flexibility of the model, algorithms have been presented that improve the option interruption \citep{mann_time-regularized_2014} or learn better options through different representations \citep{bacon_option-critic_2016, machado_temporal_2021, sorg_linear_2010}, but also research has been conducted on the cost of using options \citep{solway_optimal_2014, botvinick_hierarchically_2009}, all of which helps expanding our overall understanding of hierarchical reinforcement learning.

Another popular framework are goal-conditionned hierarchies \citep{nachum_data-efficient_2018}, in which the strategy consists in making the high-level decision of a long-term goal; then solving the problem comes down to learning how to reach the goal efficiently. In that last paper, the algorithm HIRO works on a two-level learning: the higher-level policy gets the feedback, and then send a high-level goal to the lower-level policy, along with an intrinsic reward. Hence the learning is done one personalised feedback; and thanks to the parameterisation of the sub-problems, the lower-level policy is then able to reuse experience from several sub-problems. The experiments validate the method.

In the options framework, that would amount to simultaneous option and intra-option learning. \citep{bacon_option-critic_2016} explores the idea in that context: improving parameterised options with policy gradient alongside the policy over options. An interesting point is that termination conditions are also learned; and during experiments, the authors noticed that learned options tended to stop earlier and earlier. Indeed, reducing options to the primitive actions yields a well-defined framework, which is not necessarily the case for the learned options framework; hence in \citep{harb_when_2017} a termination cost decreasing as the duration of the option increases has been introduced, to force the algorithm to learn long options and thus benefiting from temporal abstraction perks.

Finally, in \citep{fruit_exploration--exploitation_2017} a model-based approach for SMDP, directly derived from Upper-Confidence RL \citep{jaksch_near-optimal_2010}, is presented. The particularly interesting contribution of this work is that it provides the first regret analysis with options, and an insightful discussion about what makes a good option framework. The experiments reveal that preserving the average reward and the distance between pairs of states are keys to a successful temporal abstraction. That observation is compatible with the short-lived options in \citep{bacon_option-critic_2016}.

\section{Properties of Temporal Abstraction}

In this section, we will highlight the studies of temporal abstraction done in \citep{jong_utility_2008,nachum_why_2019}. That should give a first answer to the questions raised in the previous section about the properties of such approach.

\citep{jong_utility_2008} aims at discovering whether using options comes down to augmenting the core MDP (i.e. giving more possibilities to the agent) or on the contrary abstracting it, ignoring parts of the state space. Multiple experiments are conducted, on a basic GridWorld isntance and Q-Learning. A first study reveals that using options is similar to using Experience Replay (a trick to use samples multiple times): it concentrates experience around some specific trajectories, that then get more updates, alike what is done in Experience Replay. And these trajectories can be naturally identified by the agent without options. Hence, temporal abstraction isn't more sample efficient. The other aspect explored in the paper is the effect of options on exploration. It is clear that using options instead of primitive actions can significantly lower the problem size, which in turns accelerate learning, however entirely removing primitive actions can be very bad. Indeed, whenever the subgoals of the options do not exactly correspond to the most profitable state, then the agent loses a massive amount of time trying to get there, even though the option policies provided are optimal for the subtasks. In other words, the option framework must be so that not only the subtasks are solved efficiently, but also the subtasks are relevant. Otherwise the time needed for the agent to get to the goal can be highly increased, and the performance is deeply hurt. Thus, it is good to limit the availability of subtasks on some cases. This connects with the notion of \emph{diameter} for options introduced in \citep{fruit_exploration--exploitation_2017}, the shortest longest path between two states. To conclude, an option framework acts as a prior knowledge, leading to an exploration bias that, if ill-defined, can hurt more than help.

On the other hand, \citep{nachum_why_2019} is an experimental verification or reject of what the authors identified as the 4 hypotheses of the benefits of hierarchy: future rewards are propagated back faster, explored states are sort of clustered by relevance, the actions taken are all relevant towards future rewards, and finally non important actions aren't explored, so that the agent can quickly focus on profitable, high-level actions. Four environments are considered, with three different hierarchical algorithms compared to a classical agent trained on the same experience. A first experimental setting consists in exploring with varying option durations, and separately updating the high-level policy with several transition lengths. Thus the first two hypotheses are studied, one for exploration and the other for learning. The results indicate that learning from temporally extended transitions can help, however the benefits are quickly capped for an increasing duration. Regarding exploration, it indeed seem to be improved. Then, a second set of experiments is carried out in which only hierarchical learning is showcased, and not exploration. The results draw the conclusion that the impact of learning on options is negligible, the benefits actually come from learning from multi-step rewards, which does not need a specific non-hierarchical agent to train form. The last experiments are conducted in order to see whether the sample from hierarchical exploration are better. Two exploration strategies are compared, and different agents trained with this experience. In the end, the results show that indeed, the exploration is better, as non-hierarchical agents can achieve similar results than hierarchical agents with it.

Overall, those two papers give a good overview of the difficulties encountered when designing and anlysing hierarchical methods. It is also interesting to note how those works are based on experiments, as, once again, it appears that the theoretical understandings are lacking for this field of research.

\chapter{Sample-efficient learning in SMDPs}

\label{Chapter4}



\section{Sample complexity of Q-learning for SMDPs}

Algorithm, adaptation of the sample-efficiency proof for SMDPs
 
\chapter{Conclusion}
\label{Chapter5}

Going for temporal abstraction for Reinforcement Learning can be a tricky decision to take, as its effects on the training and exploration are not fully understood, leading to risks of worsening performances, contrarily to robust, reliable methods already present in classical RL. This observation holds even though many approaches have already been proposed and successfully validated by experiments, because experiments do not cover all the possible cases, and thus give no real guarantee. That is the role of theoretical analysis, of performance measures; there already exist plenty for classical algorithms, however they are lacking in the case of temporal abstraction. Hence this thesis, by providing a sample complexity bound for the most famous temporally abstracted method, makes a significant contribution to this field and gives more insight about its perks. Hopefully this is a stepping stone towards a deeper understanding of temporal abstraction. Notably, one question has been answered but several have been raised, all leading to this final goal.
 
%\include{Chapters/Chapter0}

%----------------------------------------------------------------------------------------
%	THESIS CONTENT - APPENDICES
%----------------------------------------------------------------------------------------

\appendix % Cue to tell LaTeX that the following "chapters" are Appendices

% Include the appendices of the thesis as separate files from the Appendices folder
% Uncomment the lines as you write the Appendices

% Appendix A

\chapter{Chapter 1} % Main appendix title

\label{AppendixA} % For referencing this appendix elsewhere, use \ref{AppendixA}

\section{Section 1}

Write appendix A here.

%\include{Appendices/AppendixB}
%\include{Appendices/AppendixC}

%----------------------------------------------------------------------------------------
%	BIBLIOGRAPHY
%----------------------------------------------------------------------------------------

\printbibliography[heading=bibintoc]

%----------------------------------------------------------------------------------------

\end{document}  
